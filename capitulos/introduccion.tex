El siguiente trabajo es el resultado de las notas elaboradas para el \textbf{Seminario de Geometr\'ia de la Informaci\'on, Semestre 2023-2}, impartido en el Instituto de Investigaciones en Matem\'aticas Aplicadas y Sistemas. Estas notas fueron creadas como trabajo de servicio social, dentro del programa PAPIIT-IA102823, bajo la asesor\'ia del Dr. Alessandro Bravetti (IIMAS).  
\\

La primera parte de este seminario, impartida por el Dr. Mario D\'iaz (IIMAS), nos presenta una definici\'on formal de \textbf{$f$-divergencias} \cite{1571417125811646464}, en el caso discreto. A lo largo de esta parte, se presentan distintos ejemplos y resultados sobre $f$-divergencias, con el objetivo de presentar m\'etodos matem\'aticos desde una persectiva ”fundamental" para un uso responsable de la privacidad en la informaci\'on con datos a gran escala.

Se comienza por mostrar resultados b\'asicos para $f$-divergencias y se define la privacidad diferencial local. Posteriormente en esta secci\'on, se incluye una demostraci\'on de la Desigualdad de Procesamiento de Informaci\'on \cite{polyanskiy2014lecture}, el cual es uno de los resultados m\'as fundamentales en la Teor\'ia de la Informaci\'on. Por \'ultimo, se definen los Coeficientes de Contracci\'on y con ello, se procede a enunciar la F\'ormula de Dobrushin y dar una demostrac\'ion de la misma. 
\\

La segunda parte del seminario, impartida por el Dr. Alessandro Bravetti, tiene como objetivo dar una breve introducci\'on a la \textbf{Geomet\'ia de la Informaci\'on}. Para lograr un mejor entendimiento de esta rama interdisciplinaria de las matem\'aticas, se comienza por describir esta rama a partir de distintos ejes de estudio. Mientras algunos autores mencionados aqu\'i muestran un enfoque desde las aplicaciones que surgen de esta rama \cite{amari2016information,nielsen2020elementary}, algunos otros autores nos muestran un panorama te\'orico \cite{calin2014geometric,ay2017information} a partir de resultados importantes, como lo es el Teorema Fundamental de la Geometr\'ia de la Informaci\'on~\cite{nielsen2020elementary}, 
o el Teorena de Chentsov~\cite{chentsov1982statiscal}, entre otros que se incluyen en este texto.   

Esta secci\'on, se divide en tres cap\'itulos. En el primer cap\'itulo, se intenta definir y motivar la Geometr\'ia de la Informaci\'on a partir de preguntas que ser\'an respondidas a lo largo del texto. Este cap\'itulo tambi\'en incluye un breve repaso de Inferencia Estad\'istica, donde se mencionar\'an las definiciones y teoremas que m\'as usaremos en esta rama. La segunda parte, consiste en una introducci\'on a algunas herramientas de Geometr\'ia Diferencial que se van a necesitar. En el \'utimo cap\'itulo, utilizando definiciones y resultados que se mostraron en los primeros dos cap\'itulos de esta secci\'on, se muestra una definici\'on formal de una Variedad Estad\'istica, la cual representa el principal objeto de estudio en la Geometr\'ia de la Informaci\'on, y se da una exposici\'on 
del Teorema Fundamental de la Geometr\'ia de la Informaci\'on, el cual nos permite ver a una Variedad Estad\'istica como una generalizaci\'on de una Variedad 
Riemanniana.
\\

Por \'ultimo, cabe resaltar que este trabajo no podr\'a ser utilizado en un proyecto de tesis. El material en este texto se ha elaborado principalmente con el proposito de facilitar futuras consultas en estos temas y reforzar conocimientos en \'areas como Probabilidad, Estad\'istica y Geometr\'ia Diferencial, entre otras. 