\section{Recapitulaci\'on de\\
Cap\'itulos Previos}

\begin{itemize}
    \item $f\colon(0,\infty)\to\mathbb{R}$ con $f(1)=0$ y $f$ convexa.
    \item Para $P,Q\in\mathcal{P}(\mathcal{X})$, se define su \textbf{$f$-divergencia} como
    \begin{equation*}
        D_f(P\|Q)=\sum_{x\in\mathcal{X}}f\left(\frac{P(x)}{Q(x)}\right)Q(x).
    \end{equation*}
    \item \textbf{Kernel de Markov}: $K(y|x)=\mathbb{P}(Y=y|X=x)$
    \begin{equation*}
        X\to\fbox{$K$}\to Y.
    \end{equation*}
    \item \textbf{$f$-Divergencia Condicional} de $P_{Y|X}$ y $Q_{Y|X}$ dada $P_X$ se define como
    \begin{align*}
        D_f(P_{Y|X}\|Q_{Y|X}|P_X)=\sum_{x\in\mathcal{X}}P_X(x)D_f(P_{Y|X}(\cdot|x)\|Q_{Y|X}(\cdot|x))
    \end{align*}
    \item Propiedades de las $f$-divergemcias:
    \begin{enumerate}[label=(\alph*)]
        \item $D_f(P\|Q)\geq0$,
        \item $(P,Q)\mapsto D_f(P\|Q)$ es convexa,
        \item \textbf{Desigualdad del Procesamiento de Datos}:
        \begin{align*}
            X\sim P_X\to&\fbox{$K$}\to Y\sim P_Y=KP_X\\
            X'\sim Q_X\to&\fbox{$K$}\to Y'\sim Q_Y=KQ_X
        \end{align*}
        \begin{equation*}
            D_f(KP_X\|KQ_X)\leq D_f(P_X\|Q_X).
        \end{equation*}
        \end{enumerate}
        \item Ejemplos:
        \begin{enumerate}[label=(\alph*)]
            \item \textbf{Kullback-Leibler}:
            \begin{equation*}
                D_{KL}(P\|Q)=\sum_{x\in\mathcal{X}}\log\left(\frac{P(x)}{Q(x)}\right)P(x),
            \end{equation*}
            \begin{equation*}
                f(t)=t\log(t).
            \end{equation*}
            \item \textbf{Variaci\'on Total}:
            \begin{equation*}
                TV(P,Q)=\frac{1}{2}\sum_{x\in\mathcal{X}}|P(x)-Q(x)|,
            \end{equation*}
            \begin{equation*}
                f(t)=\frac{1}{2}|t-1|.
            \end{equation*}
    \end{enumerate}
\end{itemize}
\section{Algunas Consecuencias Sobre\\ Estadisticas Suficientes}
\begin{definition}
Sea $(X,Y)$ un vector aleatorio con valores en el espacio $\mathcal{X}\times\mathcal{Y}$. Si su distribuci\'on conjunta est\'a dada por $P_{(X,Y)}$ y sus distribuciones marginales son $P_X$ y $P_Y$, su \textit{\textbf{informaci\'on mutua}} es definida por:
\begin{equation*}
    I(X;Y)=D_{KL}(P_{(X,Y)}\|P_XP_Y)
\end{equation*}
\end{definition}
\begin{observation}
Dado un par\'ametro $\theta\in\Theta$, le es asociado una muestra aleatoria $
X^n$ y a trav\'es de un Kernel de Markov $K$, se le asocia un estad\'istico $T$, \textit{i.e.}, $\theta\to X^n\overset{K}{\to}T$. 
As\'i, para cualquier estad\'istico $T$ y cualquier $P_\theta\in\mathcal{P}(\Theta)$, se cumple que:
\begin{align*}
    I(\theta;T)&=D_{KL}(P_{\theta,T}\|P_\theta P_T)\\
    &=D_{KL}\left(\overset{\sim}{K}P_{\theta,X^n}\Big\|\overset{\sim}{K}(P_\theta P_{X^n})\right)\\
    &\leq D_{KL}(P_{\theta,X^n}\|P_\theta P_{X^n})\quad\text{(DPI)}\\
    &=I(\theta;X^n).
\end{align*}
\end{observation}

\begin{definition}
Una estad\'istica $T$ de una muestra $X^n$ y par\'ametro $\theta$ es \textit{\textbf{suficiente}} si 
\begin{equation*}
    I(\theta;T)=I(\theta;X^n).
\end{equation*}
\end{definition}

\section{F\'ormulas Variacionales para $TV$ y $KL$}

\begin{theorem}
\begin{equation*}
    TV(P,Q)=\sup_{E\subseteq\mathcal{X}}\{P(E)-Q(E)\}.
\end{equation*}
\end{theorem}
\begin{proof}[\textbf{Demostraci\'on}]
( $\geq$ ): Sea $E\subseteq\mathcal{X}$. Observemos que
\begin{align*}
    P(E)-Q(E^c)&=Q(E)-P(E^c), \\
    \Longrightarrow P(E)-Q(E)&=Q(E^c)-P(E^c).
\end{align*}
Por lo tanto,
\begin{align*}
    P(E)-Q(E)&=\frac{1}{2}\Bigl(P(E)-Q(E)+Q(E^c)-P(E^c)\Bigl)\\
    &=\frac{1}{2}\left(\sum_{x\in E}\{P(x)-Q(x)\}+\sum_{x\in E^c}\{Q(x)-P(x)\}\right)\\
    &\leq\frac{1}{2}\left(\sum_{x\in E}|P(x)-Q(x)|+\sum_{x\in E^c}|P(x)-Q(x)|\right).
\end{align*}
Por lo que,
\begin{align*}
    P(E)-Q(E)\leq\frac{1}{2}\sum_{x\in\mathcal{X}}|P(x)-Q(x)|=TV(P,Q).
\end{align*}
Puesto que $E\subseteq\mathcal{X}$ es arbitrario,
\begin{equation*}
    \sup_{E\subseteq\mathcal{X}}\{P(E)-Q(E)\}\leq TV(P,Q).
\end{equation*}
\noindent( $\leq$ ): Definimos $E_0=\{x\colon P(x)\geq Q(x)\}$. Con esta notaci\'on:
\begin{align*}
    \sup_{E\subseteq\mathcal{X}}\{P(E)-Q(E)\}&\geq P(E_0)-Q(E_0)\\
    &=\frac{1}{2}\Bigl(P(E_0)-Q(E_0)+Q(E_0^c)-P(E_0^c)\Bigl)\\
    &=\frac{1}{2}\left(\sum_{x\in E_0}\{P(x)-Q(x)\}+\sum_{x\in E_0^c}\{Q(x)-P(x)\}\right)\\
    &=\frac{1}{2}\left(\sum_{x\in E_0}|P(x)-Q(x)|+\sum_{x\in E_0^c}|P(x)-Q(x)|\right)\\
    &=TV(P,Q).
\end{align*}
\end{proof}

\begin{theorem}
\begin{equation*}
    TV(P,Q)=\frac{1}{2}\sup_{\|f\|_{\infty}\leq1}\biggl\{\sum_{x}f(x)P(x)-\sum_{x}f(x)Q(x)\biggl\},
\end{equation*}
donde $\|f\|_{\infty}=\sup_{x\in\mathcal{X}}|f(x)|$.
\end{theorem}
\begin{proof}[\textbf{Demostraci\'on}]
( $\leq$ ): Para cada $E\subseteq\mathcal{X}$, definimos la funci\'on,
\begin{equation*}
    f_E(x)=\mathds{1}_E(x)-\mathds{1}_{E^c}(x).
\end{equation*}
Es f\'acil ver que $\|f\|_{\infty}=1$ y 
\begin{align*}
    \sum_{x\in\mathcal{X}}f_E(x)P(x)&=\sum_{x\in E}P(x)-\sum_{x\in E^c}P(x)\\
    &=P(E)-P(E^c).
\end{align*}
Por la proposici\'on anterior
\begin{align*}
    TV(P,Q)&=\sup_{E\subseteq\mathcal{X}}\{P(E)-Q(E)\}\\
    &=\sup_{E\subseteq\mathcal{X}}\frac{1}{2}\Bigl(P(E)-Q(E)+Q(E^c)-P(E^c)\Bigl)\\
    &=\sup_{E\subseteq\mathcal{X}}\frac{1}{2}\left(\sum_{x\in\mathcal{X}}f_E(x)P(x)-\sum_{x\in\mathcal{X}}f_E(x)Q(x)\right)\\
    &=\sup_{f_E\colon E\subseteq\mathcal{X}}\frac{1}{2}\biggl\{\sum_{x\in\mathcal{X}}f_E(x)P(x)-\sum_{x\in\mathcal{X}}f_E(x)Q(x)\biggl\}\quad\text{( $\|f\|_\infty=1$ )}\\
    &\leq\sup_{\|f\|_\infty\leq1}\frac{1}{2}\biggl\{\sum_{x\in\mathcal{X}}f(x)P(x)-\sum_{x\in\mathcal{X}}f(x)Q(x)\biggl\}\\
    &\quad\text{( $\{f_E\colon E\subseteq\mathcal{X}\}\subseteq\{f\colon\|f\|_\infty\leq1\}$ )}.
\end{align*}
\noindent( $\geq$ ): Sea $f$ tal que $\|f\|_\infty\leq1$. Observemos que:
\begin{align*}
    \frac{1}{2}\left(\sum_{x\in\mathcal{X}}f(x)P(x)-\sum_{x\in\mathcal{X}}f(x)Q(x)\right)&=\frac{1}{2}\sum_{x\in\mathcal{X}}f(x)(P(x)-Q(x))\\
    &\leq\frac{1}{2}\sum_{x\in\mathcal{X}}|f(x)||P(x)-Q(x)|\\
    &\leq\frac{1}{2}\sum_{x\in\mathcal{X}}|P(x)-Q(x)|=TV(P,Q).
\end{align*}
Como $f$ con $\|f\|_\infty\leq1$ es arbitraria,
\begin{equation*}
    \sup_{\|f\|_\infty\leq1}\frac{1}{2}\biggl\{\sum_{x\in\mathcal{X}}f(x)P(x)-\sum_{x\in\mathcal{X}}f(x)Q(x)\biggl\}\leq TV(P,Q).
\end{equation*}
\end{proof}

\begin{theorem}[\textbf{Donsker-Varadhan}]
    \begin{equation*}
        D_{KL}(P\|Q)=\sup_f\Biggl\{\sum_{x\in\mathcal{X}}f(x)P(x)-\log\left(\sum_{x\in\mathcal{X}}e^{f(x)}Q(x)\right)\Biggl\}
    \end{equation*}
donde el supremo es sobre todas las $f$ tales que:
\begin{equation*}
    \sum_{x\in\mathcal{X}}e^{f(x)}Q(x)<\infty.
\end{equation*}
\end{theorem}
\begin{proof}[\textbf{Demostraci\'on}]
( $\leq$ ): Sea $f_0(x)=\log(P(x)/Q(x))$. Se puede verificar que 
\begin{align*}
    D_{KL}(P\|Q)&=\sum_{x\in\mathcal{X}}f_0(x)P(x)-\log\left(\sum_{x\in\mathcal{X}}e^{f(x)}Q(x)\right)\\
    &\leq\sup_{f}\Biggl\{\sum_{x\in\mathcal{X}}f(x)P(x)-\log\left(\sum_{x\in\mathcal{X}}e^{f(x)}Q(x)\right)\Biggl\}.
\end{align*}
\noindent( $\geq$ ): Sea $f$ tal que $\sum_{x}e^{f(x)}Q(x)<\infty$. Observemos que
\begin{align*}
    &\sum_{x}f(x)P(x)-\log\left(\sum_{x'\in\mathcal{X}}e^{f(x')}Q(x')\right)\\
    &=\sum_{x}\log\left(\frac{e^{f(x)}}{\sum_{x'}e^{f(x')}Q(x')}\right)P(x)\\
    &=\sum_x\log\left(\frac{P(x)}{Q(x)}\cdot\frac{e^{f(x)}Q(x)/\sum_{x'}e^{f(x')}Q(x')}{P(x)}\right)P(x)\\
    &=\sum_x\log\left(\frac{P(x)}{Q(x)}\right)P(x)-\sum_{x}\log\left(\frac{P(x)}{e^{f(x)}Q(x)/\sum_{x'}e^{f(x')}Q(x')}\right)P(x)\\
    &\leq D_{KL}(P\|Q)\quad\left(\text{ $e^{f(x)}Q(x)/\sum_{x'}e^{f(x')}Q(x')=Q(x)$ }\right).
\end{align*}
Como $f$ es arbitraria:
\begin{equation*}
    \sup_f\Biggl\{\sum_{x}f(x)P(x)-\log\left(\sum_xe^{f(x)}Q(x)\right)\Biggl\}\leq D_{KL}(P\|Q).
\end{equation*}
\end{proof}

\begin{observation}
Notemos que:
\begin{equation*}
    \sum_{x\in\mathcal{X}}e^{\lambda f(x)}Q(x)=\mathbb{E}\left[e^{\lambda Z}\right],
\end{equation*}
con $Z=f(X)$ y $X\sim Q$.
\end{observation}

\section{Demostraci\'on de la\\
Desigualdad de Pinsker}
\begin{theorem}[\textbf{Desigualdad de Pinsker}]
    \begin{equation*}
        TV(P,Q)\leq\sqrt{\frac{D_{KL}(P\|Q)}{2}}.
    \end{equation*}
\end{theorem}
\begin{proof}[\textbf{Demostraci\'on}]
Sea $f$ tal que $\|f\|_\infty\leq1$. Para cada $\lambda>0$, por el Teorema de Donsker-Varadhan, 
\begin{align*}
    D_{KL}(P\|Q)&\geq\sum_{x}\lambda f(x)P(x)-\log\left(\sum_{x\in\mathcal{X}}e^{\lambda f(x)}Q(x)\right)\\
    &=\lambda\sum_{x}f(x)P(x)-\log\left(\mathbb{E}[e^{\lambda Z}]\right),
\end{align*}
con $Z=f(X)$ y $X\sim Q$. Observemos que
\begin{align*}
    \mathbb{E}[e^{\lambda Z}]&=\mathbb{E}[e^{\lambda(Z-\mathbb{E}(Z))}]e^{\lambda\mathbb{E}[Z]}\\
    &\leq e^{\lambda^2/2}e^{\lambda\mathbb{E}[Z]}\quad\text{(Desigualdad de Hoeffding)}.
\end{align*}
Por lo cual,
\begin{equation*}
   D_{KL}(P\|Q)\geq\lambda\sum_{x}f(x)P(x)-\frac{\lambda^2}{2}-\lambda\mathbb{E}[Z]. 
\end{equation*}
Notemos que:
\begin{equation*}
    \mathbb{E}[Z]=\sum_{x}f(x)Q(x),
\end{equation*}
por lo que
\begin{equation*}
    D_{KL}(P\|Q)\geq\lambda\left(\sum_xf(x)P(x)-\sum_xf(x)Q(x)\right)-\frac{\lambda^2}{2}.
\end{equation*}
Tomando $\lambda=\sqrt{2 D_{KL}(P\|Q)}$, se concluye que:
\begin{equation*}
    \frac{1}{2}\left(\sum_xf(x)P(x)-\sum_xf(x)Q(x)\right)\leq\sqrt{\frac{D_{KL}(P\|Q)}{2}},
\end{equation*}
puesto que $f$ con $\|f\|_\infty\leq1$ es arbitraria. 
\end{proof}

\begin{corollary}
Si $P_n$, $Q$ son tales que $D_{KL}(P_n\|Q)\to0$. entonces
\begin{equation*}
    TV(P_n,Q)\to0.
\end{equation*}
\end{corollary}

\begin{observation}
Notemos que, intuitivamente, la Desigualdad de Pinsker, nos dice que, para divergencias pequeñas, las estimaciones suelen ser casi iguales.      
\end{observation}