\section{Grandes Vol\'umenes de Datos,\\ Grandes Esc\'andalos}

\begin{itemize}
    \item Ignorar la privacidad puede ser problem\'atico.
    \item Distintos escenarios, distintos significados precisos.  
\end{itemize}

\section{Privacidad en Encuestas}

\subsection{Mecanismo de Respuesta Aleatoria}
\begin{enumerate}
    \item Arroje una moneda.
    \item Si se obtiene "\'Aguila" (A), responda honestamente.
    \item Si se obtiene "Sol" (S), entonces arroje la moneda y responda:
    \begin{center}
        \text{"S\'i" si se obtiene Aguila, "No" si se obtiene Sol.}
    \end{center}
\end{enumerate}

\begin{definition}\cite{warner1965randomized}
Para $\varepsilon\geq0$, el \textit{\textbf{mecanismo de respuesta aleatoria}} se define como 
\begin{equation*}
    K(\cdot|0)=\text{Ber}(1-p_\varepsilon)\quad\&\quad K(\cdot|1)=\text{Ber}(p_\varepsilon),
\end{equation*}
donde $p_\varepsilon=e^\varepsilon/(1+e^{\varepsilon})$.
\end{definition}

\section{Privacidad Diferencial Local}

\begin{notation}\cite{evfimievski2003limiting}
Recordemos que $K(A|x)=\mathbb{P}(Y\in A|X=x)$ donde 
\begin{equation*}
    X\rightarrow\fbox{\quad$K$\quad}\rightarrow Y.
\end{equation*}
\end{notation}

\begin{definition}\cite{evfimievski2003limiting}
    Sea $\varepsilon\geq0$ y $\delta\in[0,1]$. Un kernel $K$ es \textit{\textbf{$(\varepsilon,\delta)$-PDL}} (Privacidad Diferencial Local) si para toda $x_1,x_2\in\mathcal{X}$ y $A\subset\mathcal{Y}$,
    \begin{equation*}
        K(A|x_1)\leq e^\varepsilon K(A|x_2)+\delta.
    \end{equation*}
\end{definition}

\begin{itemize}
    \item Para pequeñas $\varepsilon$ y $\delta\in[0,1]$, 
    \begin{equation*}
        |K(y|x_1)-K(y|x_2)|\lesssim\varepsilon+\delta.
    \end{equation*}
    \item La verosimilitud de $y$ dado $x$ no depende mucho de $x$. 
\end{itemize}

\begin{example}\cite{evfimievski2003limiting}
El mecanismo de respuesta aleatoria es $(\varepsilon,0)$-PDL.
\end{example}

\section{Tres Problemas Arquet\'ipicos\\ en An\'alisis de Privacidad}

\begin{itemize}
    \item Entender privacidad desde una persectiva "fundamental". 
    \item Evaluar el costo de privacidad en aplicaciones estad\'isticas espec\'ificas.
    \item Calcular/Estimar los par\'ametros de privacidad de mecanismos espec\'ificos. 
\end{itemize}

\section{PDL como Contracci\'on\\ de la HS-Divergencia}

\begin{observation}\cite{asoodeh2020contraction} Seg\'un lo que definimos en el cap\'itulo anterior, tenemos que:
\begin{itemize}
    \item Representaci\'on de la HS-Divergencia:
    \begin{equation*}
        E_\gamma(P\|Q)=\sup_{A\subset\mathcal{X}}[P(A)-\gamma Q(A)].
    \end{equation*}
    \item F\'ormula Generalizada de Dobrushin:
    \begin{equation*}
        \eta_\gamma(K)=\sup_{x_1,x_2\in\mathcal{X}}E_\gamma(K(\cdot|x_1)\|K(\cdot|x_2)).
    \end{equation*}
\end{itemize}
\end{observation}

\begin{theorem}\cite{asoodeh2020contraction}
Un mecanismo de privacidad $K$ es $(\varepsilon,\delta)$-PDL si y solo si
\begin{equation*}
    \eta_{e^{\varepsilon}}(K)\leq\delta.
\end{equation*}
\end{theorem}
\begin{proof}[\textbf{Demostraci\'on}]
\begin{align*}
    \text{$K$ es $(\varepsilon,\delta)$-PDL}&\iff\sup_{x_1,x_2\in\mathcal{X}}\sup_{A\subset\mathcal{Y}}\left[K(A|x_1)-e^\varepsilon K(A|x_2)\right]\leq\delta\\
    &\iff\sup_{x_1,x_2\in\mathcal{X}}E_{e^\varepsilon}(K(\cdot|x_1)\|K(\cdot|x_2))\leq\delta\\
    &\iff\eta_{e^\varepsilon}(K)\leq\delta.
\end{align*}
\end{proof}

\begin{theorem}[\textbf{Duchi et al. '13}]\cite{asoodeh2020contraction}
Si $K$ es $(\varepsilon,0)$-PDL, entonces
\begin{equation*}
    D_{KL}(KP\|KQ)\leq2(e^\varepsilon-1)^2\|P-Q\|_{TV}.
\end{equation*}
\end{theorem}

\begin{center}
    \textit{"Nuestra t\'ecnica principal...muestra que,\\ aplicando esquemas de muestreo de privacidad diferencial\\
    actua escencialmente como contracci\'on de distribuciones."}
\end{center}

\section{Prueba de Hip\'otesis (Binaria)}
\noindent\textbf{Meta.} De una muestra de $X$ elija entre
\begin{align*}
    &\text{Hip\'otesis Nula}        &H_0: X\sim P.\\
    &\text{Hip\'otesis Alternativa} &H_1:X\sim Q.
\end{align*}

\noindent\textbf{Prueba.} Funci\'on aleatoria $\Psi\colon\mathcal{X}\to\{0,1\}$
\begin{itemize}
    \item Taza Negativa Verdadera
    \begin{equation*}
        \alpha(\Psi):=\mathbb{P}(\Psi(X)=0|H_0).
    \end{equation*}
    \item Taza Negativa Falsa
    \begin{equation*}
        \beta(\Psi):=\mathbb{P}(\Psi(X)=0|H_1).
    \end{equation*}
\end{itemize}

\begin{observation}\cite{polyanskiy2014lecture}
    $\beta_\alpha(P,Q):=\displaystyle{\inf_{\Psi\colon\alpha(\psi)\geq\alpha}\beta(\Psi)}$.
\end{observation}

\section{El costo de PDL en la\\
Prueba de Hip\'otesis Binaria}
\begin{notation}\cite{polyanskiy2014lecture,asoodeh2020contraction}
    Sea $P^{\otimes n}$ la distribuci\'on de 
    \begin{equation*}
        X_1,X_2,\dots,X_n\overset{iid}{\sim}P.
    \end{equation*}
\end{notation}

\begin{lemma}[\textbf{Stein}]\cite{polyanskiy2014lecture,asoodeh2020contraction}
Si $\alpha\in(0,1)$, entonces
\begin{equation*}
    \varliminf _{n\rightarrow \infty }\frac{1}{n}\log\left(\frac{1}{\beta_\alpha(P^{\otimes n},Q^{\otimes n})}\right)=D_{KL}(P\|Q).
\end{equation*}
\end{lemma}
\begin{observation}\cite{polyanskiy2014lecture,asoodeh2020contraction}
Recordemos los siguientes puntos
    \begin{enumerate}[label=(\alph*)]
        \item Bajo $(\varepsilon,\delta)$-PDL: $P\to KP$, $Q\to KQ$ \&
        \begin{equation*}
            D_{KL}(KP\|KQ)\leq\eta_{KL}(K)D_{KL}(P\|Q).
        \end{equation*}
        \item Por la relaci\'on entre $\eta_f$ y $\eta_\gamma$,
        \begin{equation*}
            \eta_{KL}(K)\leq1-\frac{1-\eta_{e^{\varepsilon}}(K)}{e^{\varepsilon}}.
        \end{equation*}
        \item Dado que $K$ es $(\varepsilon,\delta)$-PDL $\iff$ $\eta_{e^{\varepsilon}}(K)\leq\delta$,
        \begin{equation*}
            \eta_{KL}(K)\leq1-\frac{1-\delta}{e^\varepsilon}.
        \end{equation*}
    \end{enumerate}
\end{observation}

\begin{theorem}\cite{polyanskiy2014lecture,asoodeh2020contraction}
    Si $K$ es $(\varepsilon,\delta)$-PDL, entonces
    \begin{equation*}
        \varliminf_{n\to\infty}\frac{1}{n}\log\left(\frac{1}{\beta_\alpha(KP^{\otimes n},KQ^{\otimes n})}\right)\leq\left(1-\frac{1-\delta}{e^\varepsilon}\right)D_{KL}(P\|Q).
    \end{equation*}
\end{theorem}
\section{Privacidad Diferencial (PD)\\
bajo Composici\'on:\\
Algoritmos Iterativos}

\begin{equation*}
    \mu_0\sim W_0\to\fbox{\fbox{$K_1$}$\to$\fbox{$K_2$}$\to\cdots\to$\fbox{$K_n$}}\to W_n\sim\mu_n
\end{equation*}

\begin{itemize}
    \item PD para algoritmos iterativos:
    \begin{equation*}
        \sum_{\mu_0,\mu_0'\in\mathcal{P}(\mathcal{X})}E_{\gamma}(\mu_n\|\mu_n')\leq\delta\quad\text{con}\quad\gamma=e^\varepsilon.
    \end{equation*}
    \item Filosof\'ia de Contracci\'on:
    \begin{equation*}
        E_\gamma(\mu_n\|\mu_n')\leq\eta_{E_\gamma}(K_n)\cdot\eta_{E_\gamma}(K_1).
    \end{equation*}
\end{itemize}

\begin{example}[\textbf{Descenso del Gradiente Estoc\'astico}]
\begin{equation*}
    W_{t+1}=W_t-\eta\nabla f_t(W_t)+\sigma Z_{t+1}
\end{equation*}
\begin{itemize}
    \item $\eta,\sigma>0$,
    \item $t_1,\dots,t_n$ "Funciones de P\'erdida",
    \item $Z_1,\dots,Z_n$ \textit{i.i.d.} normales.
\end{itemize}
\end{example}

\section{\'Ultimas Observaciones}

Seg\'un la teor\'ia que hemos desarrollado, podemos decir que los problemas arquet\'ipicos en el an\'alisis de privacidad, ahora son los siguientes:

\begin{itemize}
    \item Entender privacidad desde una persectiva "fundamental". 
    \item Evaluar el costo de privacidad en aplicaciones estad\'isticas espec\'ificas.
    \item Calcular/Estimar los par\'ametros de privacidad de mecanismos espec\'ificos. 
    \item Proporcionar evidencia estadistica de implementaci\'on incorrecta. 
\end{itemize}
